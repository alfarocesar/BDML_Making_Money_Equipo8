\documentclass[12pt,a4paper,onecolumn]{article}

%%%%%%%%%%%%%%%%%%%%%%%%%%%%%%%%%%%
% PAQUETES
%%%%%%%%%%%%%%%%%%%%%%%%%%%%%%%%%%%

\usepackage[margin=1in]{geometry}
\usepackage{authblk}
\usepackage[utf8]{inputenc}  % UTF-8 evita problemas de caracteres
\usepackage[T1]{fontenc}     % Mejor soporte de fuentes en LaTeX
\usepackage[spanish]{babel}  % Manejo correcto de idioma español
\usepackage{amsfonts}
\usepackage{graphicx} % Necesario para incluir imágenes
\usepackage{xcolor}
\usepackage{amsmath}
\usepackage{amssymb}
\usepackage[table]{xcolor}
\usepackage{setspace}
\usepackage{booktabs}
\usepackage{dcolumn}
\usepackage{rotating}
\usepackage{threeparttable}
\usepackage[capposition=top]{floatrow}
\usepackage[labelsep=period]{caption}
\usepackage{subcaption}
\usepackage{multicol}
\usepackage[bottom]{footmisc}
\usepackage{enumerate}
\usepackage{units}
\usepackage{placeins}
\usepackage{booktabs,multirow}
\usepackage{float}
\usepackage{pdflscape}      % Para landscape completo
\usepackage{lscape}         % Alternativa si pdflscape da problemas
\usepackage{longtable}      % Para tablas que ocupan más de una página
\usepackage{geometry}       % Para ajustar márgenes si es necesario

% Bibliografía
\usepackage{natbib}
\bibliographystyle{apalike}
\bibpunct{(}{)}{,}{a}{,}{,}

% Formato de párrafos
\renewcommand{\baselinestretch}{1}

% Definir columnas para tablas
\usepackage{array}
\newcolumntype{L}[1]{>{\raggedright\let\newline\\\arraybackslash\hspace{0pt}}m{#1}}
\newcolumntype{C}[1]{>{\centering\let\newline\\\arraybackslash\hspace{0pt}}m{#1}}
\newcolumntype{R}[1]{>{\raggedleft\let\newline\\\arraybackslash\hspace{0pt}}m{#1}}

\usepackage{xfrac}
\usepackage{bbold}

\setcounter{secnumdepth}{6}

\usepackage{titlesec}
\titleformat*{\subsection}{\normalsize \bfseries}

\usepackage[colorlinks=true,linkcolor=black,urlcolor=blue,citecolor=blue]{hyperref}

%%%%%%%%%%%%%%%%%%%%%%%%%%%%%%%%%%%
%     TÍTULO, AUTORES Y FECHA              %
%%%%%%%%%%%%%%%%%%%%%%%%%%%%%%%%%%%

\title{\textbf{Taller 3 - Making Money with Machine Learning}}

\author{%
\begin{center}
EQUIPO 8:\\
Harold Stiven Acuña\\
José David Cuervo\\
José David Dávila\\
César Augusto Alfaro
\end{center}%
}

\date{\today}

% Configuración simple para espaciado de párrafos
\setlength{\parskip}{0.6em} % Espacio entre párrafos
\setlength{\parindent}{1em} % Sangría moderada

\begin{document}

\maketitle
\thispagestyle{empty}

%%%%%%%%%%%%%%%%%%%%%%%%%%%%%%%%%%%
% ABSTRACT
%%%%%%%%%%%%%%%%%%%%%%%%%%%%%%%%%%%

\begin{abstract}
Texto del abstract
\end{abstract}

\medskip

\begin{flushleft}
    {\bf Palabras clave:} precio de propiedades, aprendizaje automático \\
    {\bf Clasificación JEL:} J31, C53, J16
\end{flushleft}

% Añadir información del repositorio GitHub
\begin{center}
    \textit{Repositorio GitHub:} \url{https://github.com/alfarocesar/BDML_Making_Money_Equipo8}
\end{center}

\pagebreak
\singlespacing
\setlength{\parindent}{0pt}
\setlength{\parskip}{1em}

%%%%%%%%%%%%%%%%%%%%%%%%%%%%%%%%%%%
%           DOCUMENTO                       %
%%%%%%%%%%%%%%%%%%%%%%%%%%%%%%%%%%%

\section{Introducción}

Incluir la introducción

\section{Datos}

\subsection{Adecuación de los datos}

\subsection{Construcción de la muestra}

\subsubsection{Análisis y limpieza de las bases de datos}

\subsubsection{Transformación de variables de personas}

\subsubsection{Agregación de datos a nivel de hogar}

\subsection{Limpieza de datos y tratamiento de valores faltantes}

\subsection{Análisis descriptivo}

\subsubsection{Distribución de variables numéricas}

\subsubsection{Análisis de outliers}

\subsubsection{Relación con la variable objetivo}

\subsection{Selección final de variables}

\section{Modelos y Resultados}

\subsection{Modelo de Selección y Entrenamiento}

\subsubsection{Metodología}

\subsubsection{Variables utilizadas}

\subsection{Resultados}


\subsection{Matrices de Confusión}


\section{Conclusión}


%%%%%%%%%%%%%%%%%%%%%%%%%%%%%%%%%%%
% ANEXOS
%%%%%%%%%%%%%%%%%%%%%%%%%%%%%%%%%%%

\clearpage
\appendix
\section*{Anexos}
\addcontentsline{toc}{section}{Anexos}

%\section{Estadísticas descriptivas de variables continuas}
%\input{../views/tables/01_descriptiva_personas_continua.tex}

%\section{Estadísticas descriptivas de variables discretas}
%\input{../views/tables/02_descriptiva_personas_discreta.tex}


%%%%%%%%%%%%%%%%%%%%%%%%%%%%%%%%%%%
% TERMINA EL CONTENIDO
%%%%%%%%%%%%%%%%%%%%%%%%%%%%%%%%%%%

\pagebreak
\singlespacing
\nocite{*}
\bibliographystyle{apalike}
\bibliography{references}
\end{document}

%%%%%%%%%%%%%%%%%%%%%%%%%%%%%%%%%%%
% TERMINA EL DOCUMENTO
%%%%%%%%%%%%%%%%%%%%%%%%%%%%%%%%%%%
